\halftitle{$title$}{}

\textit{An Introduction to Quantitative Text Analysis for Linguistics: Reproducible Research Using R} is a pragmatic textbook that equips students and researchers with the essential concepts and practical programming skills needed to conduct quantitative text analysis in a reproducible manner. Designed for undergraduate students and those new to the field, this book assumes no prior experience with statistics or programming, making it an accessible resource for anyone embarking on their journey into quantitative text analysis.


Through a pedagogical approach which emphasizes intuitive understanding over technical details, readers will gain data literacy by learning to identify, interpret, and evaluate data analysis procedures and results. They will also develop research skills, enabling them to design, implement, and communicate quantitative text analysis projects effectively. The book places a strong emphasis on programming skills, guiding readers through interactive lessons, tutorials, and lab activities using the R programming language and real-world datasets.


This practical text is enriched with features that facilitate learning, including thought and practical exercises, a companion website which includes programming demonstrations to develop and augment readers' recognition of how programming strategies are implemented, and a GitHub repository which contains both a set of interactive R programming lessons (Swirl) and lab exercises, which guide readers through practical hands-on programming applications. This text is an essential companion to any linguist looking to learn how to incorporate quantitative data analysis into their work.


\textbf{Jerid Francom} is Associate Professor of Spanish and Linguistics at Wake Forest University. His research focuses on the use of language corpora from a variety of sources (news, social media, and other internet sources) to better understand the linguistic and cultural similarities and differences between language varieties for both scholarly and pedagogical projects. He has published on topics including the development, annotation, and evaluation of linguistic corpora and analyzed corpora through corpus, psycholinguistic, and computational methodologies. He also has experience working with and teaching statistical programming with R.

$if(has-frontmatter)$
\frontmatter
$endif$
$if(title)$
$if(beamer)$
\frame{\titlepage}
$else$
\maketitle
$endif$
$if(abstract)$
\begin{abstract}
$abstract$
\end{abstract}
$endif$
$endif$
